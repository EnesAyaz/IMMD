\documentclass[conference,a4paper,twocolumn]{IEEEtran}
\usepackage{graphicx}
\graphicspath{{images/}}
\usepackage{ifpdf}
\usepackage{cite}
\ifCLASSINFOpdf
  % \usepackage[pdftex]{graphicx}
  % declare the path(s) where your graphic files are
  % \graphicspath{{../pdf/}{../jpeg/}}
  % and their extensions so you won't have to specify these with
  % every instance of \includegraphics
  % \DeclareGraphicsExtensions{.pdf,.jpeg,.png}
\else
  % or other class option (dvipsone, dvipdf, if not using dvips). graphicx
  % will default to the driver specified in the system graphics.cfg if no
  % driver is specified.
  % \usepackage[dvips]{graphicx}
  % declare the path(s) where your graphic files are
  % \graphicspath{{../eps/}}
  % and their extensions so you won't have to specify these with
  % every instance of \includegraphics
  % \DeclareGraphicsExtensions{.eps}
\fi

\usepackage[cmex10]{amsmath}
\interdisplaylinepenalty=2500

\usepackage{algorithmic}
\usepackage{array}
\usepackage{mdwmath}
\usepackage{mdwtab}
\usepackage{eqparbox}
\usepackage[tight,footnotesize]{subfigure}
\usepackage[caption=false]{caption}
\usepackage[font=footnotesize]{subfig}
\usepackage{fixltx2e}


%\usepackage{stfloats}
% stfloats.sty was written by Sigitas Tolusis. This package gives LaTeX2e
% the ability to do double column floats at the bottom of the page as well
% as the top. (e.g., "\begin{figure*}[!b]" is not normally possible in
% LaTeX2e). It also provides a command:
%\fnbelowfloat
% to enable the placement of footnotes below bottom floats (the standard
% LaTeX2e kernel puts them above bottom floats). This is an invasive package
% which rewrites many portions of the LaTeX2e float routines. It may not work
% with other packages that modify the LaTeX2e float routines. The latest
% version and documentation can be obtained at:
% http://www.ctan.org/tex-archive/macros/latex/contrib/sttools/
% Documentation is contained in the stfloats.sty comments as well as in the
% presfull.pdf file. Do not use the stfloats baselinefloat ability as IEEE
% does not allow \baselineskip to stretch. Authors submitting work to the
% IEEE should note that IEEE rarely uses double column equations and
% that authors should try to avoid such use. Do not be tempted to use the
% cuted.sty or midfloat.sty packages (also by Sigitas Tolusis) as IEEE does
% not format its papers in such ways.

% *** PDF, URL AND HYPERLINK PACKAGES ***
%
%\usepackage{url}
% url.sty was written by Donald Arseneau. It provides better support for
% handling and breaking URLs. url.sty is already installed on most LaTeX
% systems. The latest version can be obtained at:
% http://www.ctan.org/tex-archive/macros/latex/contrib/misc/
% Read the url.sty source comments for usage information. Basically,
% \url{my_url_here}.

% *** Do not adjust lengths that control margins, column widths, etc. ***
% *** Do not use packages that alter fonts (such as pslatex).         ***
% There should be no need to do such things with IEEEtran.cls V1.6 and later.
% (Unless specifically asked to do so by the journal or conference you plan
% to submit to, of course. )


% correct bad hyphenation here
\hyphenation{op-tical net-works semi-conduc-tor}


\begin{document}
\title{Elimination of the DC Bus Sixth Harmonic Component in Integrated Modular Motor Drives Using Third Harmonic Injection Method}
\author{\IEEEauthorblockN{Mesut Uğur}
\IEEEauthorblockA{Department of Electrical and Electronics Engineering\\
Middle East Technical University\\
Ankara, Turkey\\
Email: ugurm@metu.edu.tr}
\and
\IEEEauthorblockN{Ozan Keysan}
\IEEEauthorblockA{Department of Electrical and Electronics Engineering\\
Middle East Technical University\\
Ankara, Turkey\\
Email: keysan@metu.edu.tr}
}

\maketitle

\begin{abstract}
%\boldmath
In this paper, a novel method to eliminate the harmonic component occurring on the DC bus which is six times the grid frequency is proposed. This harmonic component is present due to natural commutation of the passive diode bridge rectifier in motor drive applications. In conventional drives, bulky LC filters are utilized to reduce the effect of this harmonic component to the motor drive inverter. With this method, DC bus capacitance requirement can be minimized which will enhance the power density and decrease the cost of the overall system. Third harmonic injection is used with modular inverters in an integrated modular motor drive application. Both rectifier and inverter side analytical models are presented, the elimination of the sixth harmonic component is described analytically, and verified by simulations performed on MATLAB/Simulink. The possible adverse effects of third harmonic injection method are also discussed.
\end{abstract}
\IEEEpeerreviewmaketitle



\section{Introduction}
This demo file is intended to serve as a ``starter file''
for IEEE conference papers produced under \LaTeX\ using
IEEEtran.cls version 1.7 and later.

\hfill mds
 
\hfill January 11, 2007

\subsection{Subsection Heading Here}
Subsection text here.

\subsubsection{Subsubsection Heading Here}
Subsubsection text here.


% An example of a floating figure using the graphicx package.
% Note that \label must occur AFTER (or within) \caption.
% For figures, \caption should occur after the \includegraphics.
% Note that IEEEtran v1.7 and later has special internal code that
% is designed to preserve the operation of \label within \caption
% even when the captionsoff option is in effect. However, because
% of issues like this, it may be the safest practice to put all your
% \label just after \caption rather than within \caption{}.
%
% Reminder: the "draftcls" or "draftclsnofoot", not "draft", class
% option should be used if it is desired that the figures are to be
% displayed while in draft mode.
%
%\begin{figure}[!t]
%\centering
%\includegraphics[width=2.5in]{myfigure}
% where an .eps filename suffix will be assumed under latex, 
% and a .pdf suffix will be assumed for pdflatex; or what has been declared
% via \DeclareGraphicsExtensions.
%\caption{Simulation Results}
%\label{fig_sim}
%\end{figure}

% Note that IEEE typically puts floats only at the top, even when this
% results in a large percentage of a column being occupied by floats.


% An example of a double column floating figure using two subfigures.
% (The subfig.sty package must be loaded for this to work.)
% The subfigure \label commands are set within each subfloat command, the
% \label for the overall figure must come after \caption.
% \hfil must be used as a separator to get equal spacing.
% The subfigure.sty package works much the same way, except \subfigure is
% used instead of \subfloat.
%
%\begin{figure*}[!t]
%\centerline{\subfloat[Case I]\includegraphics[width=2.5in]{subfigcase1}%
%\label{fig_first_case}}
%\hfil
%\subfloat[Case II]{\includegraphics[width=2.5in]{subfigcase2}%
%\label{fig_second_case}}}
%\caption{Simulation results}
%\label{fig_sim}
%\end{figure*}
%
% Note that often IEEE papers with subfigures do not employ subfigure
% captions (using the optional argument to \subfloat), but instead will
% reference/describe all of them (a), (b), etc., within the main caption.


% An example of a floating table. Note that, for IEEE style tables, the 
% \caption command should come BEFORE the table. Table text will default to
% \footnotesize as IEEE normally uses this smaller font for tables.
% The \label must come after \caption as always.
%
%\begin{table}[!t]
%% increase table row spacing, adjust to taste
%\renewcommand{\arraystretch}{1.3}
% if using array.sty, it might be a good idea to tweak the value of
% \extrarowheight as needed to properly center the text within the cells
%\caption{An Example of a Table}
%\label{table_example}
%\centering
%% Some packages, such as MDW tools, offer better commands for making tables
%% than the plain LaTeX2e tabular which is used here.
%\begin{tabular}{|c||c|}
%\hline
%One & Two\\
%\hline
%Three & Four\\
%\hline
%\end{tabular}
%\end{table}


% Note that IEEE does not put floats in the very first column - or typically
% anywhere on the first page for that matter. Also, in-text middle ("here")
% positioning is not used. Most IEEE journals/conferences use top floats
% exclusively. Note that, LaTeX2e, unlike IEEE journals/conferences, places
% footnotes above bottom floats. This can be corrected via the \fnbelowfloat
% command of the stfloats package.



\section{Analytical model of the rectifier}
Rectifier model, 6th harmonic component injection, LC filter characteristics


A conventional motor drive application block diagram is shown in Fig. \ref{fig:conv_motor_drive}. The DC link decouples the inverter and rectifier such that, its characteristics is effected from both sides. Most studies consider only one side for DC link characterisation or filter component optimization, although they should be considered simultaneously. This research aims at modeling the system as a whole, investigating the effect of harmonic components injected to the DC link from both sides and eliminating the low frequency harmonic due to the rectifier side by using the modular structure of the inverter side.

\begin{figure}[htp]
  \centering
  \includegraphics[width=8cm]{images/conv_motor_drive}
  \caption{A conventional motor drive block diagram}
  \label{fig:conv_motor_drive}
\end{figure}

Diode bridge rectifier is a natural-commutated converter, circuit schematic of which is shown in Fig. \ref{fig:rect_circuit}.

\begin{figure}[htp]
  \centering
  \includegraphics[width=8cm]{images/rect_circuit}
  \caption{Diode bridge rectifier circuit diagram}
  \label{fig:rect_circuit}
\end{figure}

A set of voltage and current waveforms are also shown in Fig. \ref{fig:rect_waveform}, for 400V line-to-line grid voltage at 50 Hz, filter inductance of 1 mH, filter capacitance of 3 mF and load resistance of 
10 $\Omega$. The three-phase rectifier output voltage and current has large harmonic components frequency of which is six times the grid frequency. This component is filtered by a second order LC filter resulting in a much smoother load voltage and current. Since the harmonic frequency is relatively low in comparison with conventional switching frequencies, large inductance and capacitance values are needed on the DC link filter. Those passive elements constitute a large portion of overall volume and cost, hence it is aimed to minimize their values.

\begin{figure}[htp]
  \centering
  \includegraphics[width=8cm]{images/rect_waveform}
  \caption{Diode bridge rectifier input and output waveforms}
  \label{fig:rect_waveform}
\end{figure}


\section{Description of the proposed method}
 Proposed method: Sixth harmonic creation on the DC link with third harmonic injection (analytical)



\section{Implementation of the method and practical issues}
New IMMD scheme for third harmonic injection, practical considerations, effects on other components, torque ripple, copper loss etc.



\section{Results}
Simulation results


\section{Conclusion}
The conclusion goes here.




% conference papers do not normally have an appendix


% use section* for acknowledgement
\section*{Acknowledgment}


The authors would like to thank...





% trigger a \newpage just before the given reference
% number - used to balance the columns on the last page
% adjust value as needed - may need to be readjusted if
% the document is modified later
%\IEEEtriggeratref{8}
% The "triggered" command can be changed if desired:
%\IEEEtriggercmd{\enlargethispage{-5in}}

% references section

% can use a bibliography generated by BibTeX as a .bbl file
% BibTeX documentation can be easily obtained at:
% http://www.ctan.org/tex-archive/biblio/bibtex/contrib/doc/
% The IEEEtran BibTeX style support page is at:
% http://www.michaelshell.org/tex/ieeetran/bibtex/
%\bibliographystyle{IEEEtran}
% argument is your BibTeX string definitions and bibliography database(s)
%\bibliography{IEEEabrv,../bib/paper}
%
% <OR> manually copy in the resultant .bbl file
% set second argument of \begin to the number of references
% (used to reserve space for the reference number labels box)
\begin{thebibliography}{1}

\bibitem{IEEEhowto:kopka}
H.~Kopka and P.~W. Daly, \emph{A Guide to \LaTeX}, 3rd~ed.\hskip 1em plus
  0.5em minus 0.4em\relax Harlow, England: Addison-Wesley, 1999.

\end{thebibliography}




% that's all folks
\end{document}


